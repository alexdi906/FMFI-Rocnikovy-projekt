\documentclass[11pt]{article}
\usepackage{algorithms}
\usepackage{amsmath}
\usepackage{amsfonts} 
\usepackage{listings}
\usepackage{geometry}
\usepackage{mathtools}
\usepackage[slovak,shorthands=off]{babel}
\usepackage{graphicx}
\usepackage{float}
\usepackage{amsthm}
\usepackage{algpseudocodex}
\usepackage[utf8]{inputenc} % Input encoding
\usepackage[T1]{fontenc}    % Font encoding for special characters
\geometry{margin=2cm}
\graphicspath{ {./images/} }
\pagenumbering{gobble}
\setlength{\parskip}{0.75\baselineskip}
\setlength{\parindent}{0pt}

\theoremstyle{definition}
\newtheorem{lemma}{Lema}
\begin{document}

\newtheorem{definition}{Definícia}

\theoremstyle{definition}
\newtheorem{theorem}{Tvrdenie}

\title{RP}
\begin{center}
    {\LARGE Ročníkový projekt zimný semester report}\\[1em]
    {\large Alex Diko  \\ \texttt{diko3@uniba.sk}}\\[1em]
\end{center}

\paragraph{Pojmy}

\textit{Graf} je dvojica $G = (V,E)$, kde $V,E$ sú množiny také, že $E \subseteq V\times V$
a $V \cap E = \emptyset$. Prvky $V$ sa nazývajú \textit{vrcholy} grafu $G$ a prvky $E$ jeho \textit{hrany}.
Množinu hrán grafu $G$ označujeme ako $E(G)$. Vrchol 
$v$ je \textit{incidentný} s hranou $e$ ak $v\in e$. \textit{Stupeň vrcholu} je počet
hrán s ním incidentných. Ak všetky vrcholy grafu
majú stupeň $k$, graf sa nazýva sa \textit{$k$-regulárny}. \textit{Cesta} je neprázdny graf $P = (V,E)$,
kde $V = \{x_0, x_1, \ldots, x_k\}$ a $E = \{x_0x_1, x_1x_2, \ldots, x_{k-1}x_k\}$,
pričom všetky $x_i$ sú rôzne. Túto cestu označíme $x_0x_1\ldots x_k$. Ak $P = x_0\ldots x_{k-1}$ je cesta, potom
graf $C := P + x_{k-1}x_0$ sa nazýva \textit{kružnica}. \textit{Dĺžka} kružnice je počet jej hrán. Graf
$ G'(V',E')$ je \textit{podgrafom} grafu $G$ ak $V' \subseteq V$ a $E' \subseteq E$.
\textit{Rozklad} množiny $E(G)$ je množina $R = \{R_1, \ldots, R_k\}$, kde $R_i$
sú navzájom disjunktné a zjednotenie $\cup R$ všetkých množín $R_i \in R$ sa rovná $E(G)$.

\textit{Kružnicovou dekompozíciou grafu} nazveme rozklad $E(G)$ na kružnice, ktoré nezdieľajú žiadnu
hranu. Ak každá kružnica tohto rozkladu má párnu dĺžku nazveme tento rozklad \textit{párnym} (ďalej
\textit{ECD} z anglického \textit{Even cycle decomposition}).
Pre ECD grafu zafarbíme každú jeho kružnicu tak, aby kružnice so spoločným
vrcholom nemali rovnakú farbu. Potom zjednotenie kružníc v každej farebnej triede 
bude 2-regulárny podgraf pôvodného grafu. Ak najmenší počet farieb potrebných 
na takéto zafarbenie je $m$ potom ECD rozklad má \textit{veľkosť} $m$.

\paragraph{Implementácia}
Do grafovej knižnice \textit{ba-graph} som naimplementoval backtracking (prehľadávanie 
s návratom) algoritmus, ktorý nájde veľkosť ECD grafu. Pseudokód algoritmu je uvedený nižšie.

\begin{algorithm}
\begin{algorithmic}[1]
    
    \Function{SkúsZačaťNovúKružnicu}
        \If{Všetky hrany sú priradené farebnej triede}
            \State Zapamätaj si toto zafarbenie a porovnaj jeho veľkosť s doteraz najmenším
        \Else
            \State $h \gets$ Zober nejakú nezafarbenú hranu
            \ForAll{$f \in$ Aktuálne farebné triedy}
                \State Skús priradiť $h$ do $f$ a zapamätaj si, že $e$ bude v kružnici na párnej pozícii
                \State SkúsPokračovaťVAktuálnejKružnici(e)
            \EndFor

            \State Skús priradiť $h$ do novej farebnej triedy a zapamätaj si, že $e$ bude v kružnici na párnej pozícii
            \If{Počet farebných tried je väčší ako aktuálne najlepšie riešenie}
            \State odmietni aktuálne riešenie
            \State SkúsPokračovaťVAktuálnejKružnici($h$)
            \EndIf
        \EndIf

    \EndFunction

    \Function{SkúsPokračovaťVAktuálnejKružnici}{\textit{aktuálnaHrana}}
        \LComment{Over či je aktuálnaHrana správne zafarbená. V korektnej ECD 
        musia byť práve 2 susedné hrany v rovnakej farebnej triede a na pozíciach 
        inej parity ako AktuálnaHrana}
        \State $p \gets 0$ \Comment{Počet susedných hrán v rovnakej farebnej triede ale na pozíciach
        rôznej parity}
        \ForAll{$h \in$ hrany susedné s \textit{aktuálnaHrana}}
        \If{$h$ je v rovnakej farebnej triede ako \textit{aktuálnaHrana}}
            \If{$h$ je na pozícii rovnakej parity ako \textit{aktuálnaHrana}}
            \State Odmietni aktuálnu kružnicu 
            \Else
            \State $p \gets p+1$
            \EndIf
        \EndIf
        \EndFor
        
        \If{$p > 2$}
        \State Odmietni aktuálnu kružnicu
        \ElsIf{$p = 2$}
        \State \Call{SkúsZačaťNovúKružnicu}{} \Comment{Bez sporu sa nám podarilo vytvoriť kružnicu párnej dĺžky}
        \Else
        \ForAll{$h \in$ hrany susedné s \textit{aktuálnaHrana}}
            \State Skús priradiť $h$ do rovnakej farebnej triedy ako aktuálnaHrana ale na pozíciu inej parity
            \Call{SkúsPokračovaťVAktuálnejKružnici}{$h$}
        \EndFor
        \EndIf
        
    \EndFunction
\end{algorithmic}
\end{algorithm}

\end{document}
